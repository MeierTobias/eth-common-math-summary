\ptitle{Orthogonal Matrix}
\noindent\begin{equation*}
    \mathbf{U}^{\mathsf{T}}\mathbf{U}=\mathbf{UU}^{\mathsf{T}}=\mathbf{I}\quad \mathbf{U}\in \mathbb{R}^{n\times n}
\end{equation*}

\ptitle{Unitary Matrix}
\noindent\begin{equation*}
    \mathbf{U}^{\mathsf{H}}\mathbf{U}=\mathbf{UU}^{\mathsf{H}}=\mathbf{I}\quad \mathbf{U}\in \mathbb{R}^{n\times n}
\end{equation*}
\begin{itemize}
    \item preserves the euclidean norm:
          \noindent\begin{equation*}
              \|\mathbf{Ux}\| _2=\|\mathbf{x}\|_2 \quad \forall \mathbf{x} \in \mathbb{C}^n
          \end{equation*}
\end{itemize}

\ptitle{Hermitian Matrix}
\noindent\begin{equation*}
    \mathbf{S}=\mathbf{S}^{\mathsf{H}}
\end{equation*}
For any
\noindent\begin{equation*}
    \mathbf{A} \in \mathbb{R}^{m\times n}
\end{equation*}
both
\noindent\begin{align*}
    \mathbf{A^H A} & \in \mathbb{R}^{n\times n} \\
    \mathbf{A A^H} & \in \mathbb{R}^{m\times m}
\end{align*}
are hermitian, positive semi-definite and their eigenvalues are real and non-negative.
\begin{itemize}
    \item For every hermitian matrix $\mathbf{S}$ exists a unitary matrix $\mathbf{U}$ s.t. $\mathbf{U}^{\mathsf{H}}\mathbf{SU}$ is a diagonal matrix.
          \begin{itemize}
              \item In other words, unitary matrices can diagonalize hermitian matrices which is what one uses for SVD.
          \end{itemize}
\end{itemize}

\newpar{}
\ptitle{Matrix Exponential Function}
\begin{equation*}
    e^{\mathbf{X}} = \sum_{k=0}^{\infty}\frac{1}{k!}\mathbf{X^k}
\end{equation*}